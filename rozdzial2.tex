\chapter{Omówienie dziedziny}
\label{cha:omowienieDziedziny}

\section{HTML5}
\label{sec:html5}
HTML5 jest kolejną wersją języka HTML służącego to tworzenia stron WWW,
opracowywany przez World Wide Web Consortium (W3C) i Web Hypertext Application
Technology Working Group (WHATWG). Standard jest w większości ukończony i jego ostateczna
specyfikacja ma być wydana do końca roku 2014, natomiast w ciągu kolejnych dwóch lat
ma nastąpić wydanie wersji 5.1. Jest on pomyślany jak następca HTML 4.1 i XHTML 1.0.

HTML5 wprowadza wiele nowych tagów (np. canvas, video, audio) oraz interfejsów programistycznych
wymienionych w dalszych podrozdziałach. 
Wraz z nowymi możliwościami HTML5 wraz z językiem JavaScript, stał się platformą
umożliwiającą tworzenie rozbudowanych aplikacji oraz gier.

\subsection{JavaScript}
\label{ssec:javascript}

JavaScript jest skryptowym językiem programowania osadzony w przeglądarkach internetowych
i wykorzystywany do tworzenia interaktywnych stron oraz aplikacji internetowych.
Formalnie nie jest związany z HTML5, jednak jest jedynym językiem obsługiwanym przez
wszystkie przeglądarki, dlatego wszystkie wymienione w dalszych podrozdziałach
interfejsy programistyczne są stworzone dla tego języka.

JavaScript powstał 1995 w firmie Netscape, a następnie został ustandaryzowany przez ECMA (stąd
stosowana formalnie nazwa ECMAScript). Nazwa i składnia przypomina Javę, jednak jest to podobieństwo
mylące. JavaScript jest językiem dynamicznym, posiadającym wiele cech języków funkcyjnych,
a obiektowość jest oparta na prototypach, a nie klasach.

Początkowo JavaScript był używany do wyświetlania prostych animacji,
komunikatów i wstępnej walidacji danych w formularzach. Z w JavaScripcie czasem zaczęły
powstawać coraz bardziej rozbudowane aplikacje i okazało się, że wydajność interpretowanego
języka skryptowego z Garbage Collectorem (odśmiecaczem) jest coraz większym problemem.
Rozpoczął się wyścig pomiędzy twórcami
przeglądarek o jak najszybszy silnik skryptowy. W rezultacie, obecnie w większości przeglądarek
skrypty są kompilowane do kodu natywnego przez kompilator JIT (Just In Time), z zastosowaniem wielu
technik optymalizacji kodu, a Garbage Collector jest coraz szybszy. Dostępne są również
rozbudowane narzędzia do profilowania kodu JavaScript.

Ta sytuacja sprawiła, że JavaScript zaczął się nadawać do robienia wymagających obliczeniowo
gier 3d. Oczywiście wciąż trzeba unikać szczególnie nieefektywnych konstrukcji tego języka
i starać się alokować jak najmniej obiektów, aby ograniczyć przestoje powodowane przez działanie
odśmiecacza. Jednak przy zachowaniu tych zasad, wydajność współczesnych silników JavaScript
powinna być wystarczająca dla wielu gier.


\subsubsection{asm.js}
\label{sec:asmjs}

asm.js jest to podzbiór JavaScript opracowany przez Mozillę, który zapewnia najszybsze wykonanie.
Programowanie zgodnie z asm.js jest jednak bardzo żmudne, dlatego podzbiór ten jest głównie
pomyślany jak cel kompilatorów innych języków do JavaScript (np. Emscripten [link/footnote]). Nie istnieje
obecnie translator dowolnego kodu JavaScript do asm.js. Ponieważ gra będąca przedmiotem niniejszej
pracy powstaje w JavaScript, tematyka asm.js nie będzie szerzej omawiana.

\subsection{WebGL}
\label{ssec:webgl}

WebGL jest to API dla języka JavaScript służące do wyświetlania grafiki 3d w przeglądarce.
Z tego powodu jest to najbardziej istotna technologia z punktu widzenia
twórców gier. Pomimo, że formalnie WebGL nie jest jeszcze
częścią standardu HTML5, to jest on zaimplementowany we wszystkich znaczących
przeglądarkach.

WebGL bazuje na standardzie OpenGL ES 2.0 (Open Graphics Library for
Embedded Systems) przeznaczonym dla urządzeń mobilnych, który z kolei
jest uproszczoną wersją OpenGL (Open Graphics Library) -- otwartego API do
renderowania grafiki 3d, będącym jednym z dwóch (obok Direct3D) szeroko
stosowanych API do wyświetlania grafiki w grach i programach wykorzystujących 3d.
Wszystkie te standardy zostały opracowane przez konsorcjum Khronos Group (wcześniej ARB),
zrzeszające wszystkie (poza Microsoftem) większe firmy zainteresowane tematem grafiki 3d.

Prace nad WebGL rozpoczęła Mozilla w 2006 roku. A w następnym roku Firefox oraz Opera
miały już pierwsze działające implementacje. Były one kontynuowane w grupie roboczej
WebGL Working Group działającej w ramach Khronos Group z udziałem m. in. Mozilli,
Opery, Google oraz Apple. W 2011 roku ukończona została pierwsza wersja standardu, a w 2013
ruszyły prace nad wersją 2.0 bazującą na OpenGL 3.0. W tym też roku Microsoft wydał
Internet Explorer 11 z obsługą WebGL, pomimo początkowej niechęci do tego standardu
(Microsoft promuje swoją technologię Direct3D, konkurencyjną w stosunku do OpenGL).
Tym samym ostatnia z liczących się przeglądarek dodała obsługę WebGL.

WebGL zaprojektowano dla języka JavaScript, w przeciwieństwie do OpenGL i OpenGL ES,
które przeznaczone są głównie dla języka C. Jednakże jest interfejsem bardzo
niskopoziomowym, w zasadzie dokładnie przełożonym na JavaScript OpenGL ES. Dodano
jedynie kilka usprawnień (jak wczytywanie tekstury z elementu HTML <img>) oraz
dodatkową walidację danych, istotną w środowisku tekstowym. Z tego powodu programiści
JavaScript mogą uznać API za trudne i nieprzystępne. Z drugiej jednak strony, taka
implementacja gwarantuje maksymalną szybkość działania, a programiści znający OpenGL mogą
w zasadzie natychmiast zacząć używać WebGL.


\subsection{Web Workers}
\label{ssec:webWorkers}

Współcześnie każdy domowy komputer posiada wielordzeniowy procesor. W tym samym kierunku
podążają procesory w urządzeniach mobilnych. Aby maksymalnie wykorzystać moc obliczeniową,
konieczne jest tworzenie aplikacji współbieżnych. Niestety JavaScript przez długi czas
tego nie umożliwiał. Zmienia to API Web Workers należące do standardu HTML5.

Worker jest to odpowiednik wątku dla języka JavaScript. Jednakże jego sposób działania
upodabnia go bardziej do procesu -- poszczególne workery nie mogą współdzielić pamięci
i komunikują się wyłącznie za pomocą wysyłanych wiadomości. Dzięki takiej konstrukcji
można uniknąć wielu problemów związanych z klasyczną wielowątkowością, jednak odbywa
się to kosztem wydajności z powodu konieczności częstego kopiowania danych. Dodatkowo
API dostępne dla workera jest bardzo ograniczone (np. nie ma dostępu do elementów HTML).

Te cechy powodują, że aby osiągnąć zysk wydajnościowy przy pomocy Web Workers, trzeba od
początku projektować aplikację pod kątem tego API -- tak aby ograniczyć ilość przesyłanych
danych.

Należy zaznaczyć, iż taki sposób tworzenia aplikacji wielowątkowych jest nieobcy programistom
gier, gdyż podobny model wymuszał procesor Cell w konsoli PlayStation3. Procesor ten składa
się z jednego głównego rdzenia (tzw. Power Processing Element) oraz kilku rdzeni wspomagających
(tzw. Synergistic Processing Element), które również nie mogą współdzielić pamięci. Podobnie jak
Web Workerze, tak w programie działającym na SPE, dostępne API jest bardzo ograniczone.

\subsection{Web Sockets}
\label{ssec:webSockets}

Wiele współczesnych gier umożliwia wspólną rywalizację wielu graczy za pośrednictwem internetu
(tzw. multiplayer). W celu implementacji gry sieciowej konieczne jest API do komunikacji
internetowej.

Interfejs Web Sockets jest odpowiednikiem systemowych gniazd sieciowych i umożliwia komunikację
w czasie rzeczywistym pomiędzy aplikacją JavaScript, a serwerem za pośrednictwem protokołu TCP.
Jest on częścią HTML5 i dostępny we wszystkich znaczących przeglądarkach.

\subsection{WebRTC}
\label{ssec:webrtc}

Wprowadzenie interfejsu Web Sockets było istotnym krokiem z punktu widzenia twórców gier
sieciowych, ma on jednak pewne ograniczenia. Jest oparty o protokół TCP, którego 
niezawodność powoduje opóźnienia w transmisji danych (przesyłanie potwierdzeń,
retransmisje itp.). Dodatkowo cały ruch musi być kierowany przez serwer, gdyż bezpośrednia
komunikacja pomiędzy przeglądarkami jest niemożliwa. To sprawia, że Web Sockets nie nadają
się do bardzo dynamicznych gier.

Rozwiązanie tego problemu nadeszło ze strony interfejsu WebRTC (od Real Time Communication),
mającego służyć przede wszystkim
do przesyłanie wideo i dźwięku w czasie rzeczywistym pomiędzy przeglądarkami, dzięki czemu
możliwe byłoby zbudowanie komunikatora wideo w przeglądarce.

WebRTC umożliwia również przesyłanie dowolnych danych tekstowych i binarnych, a ponieważ
działa w oparciu o protokół UDP i łączy bezpośrednio przeglądarki (peer-to-peer), idealnie
nadaje się do szybkich gier sieciowych. Oczywiście użycie  bezpołączeniowego protokołu UDP
powoduje, że aplikacja sama musi zapewnić poprawność działania w przypadku błędów przesyłu. 

WebRTC był początkowo tworzonym przez Google, lecz wkrótce prace przejęła grupa robocza w W3C,
złożona m.in. z Google, Mozilli i Opery. Niestety Microsoft odmówił wsparcia dla WebRTC i  -- jak
to miało miejsce już wielokrotnie -- zaczął tworzyć swój alternatywny standard: CU-RTC-WEB
(Customizable, Ubiquitous Real-Time Communication over the Web). Dlatego WebRTC nie jest
obsługiwany w Internet Explorer i nie ma żadnych planów jego implementacji. Co więcej,
żadna z przeglądarek nie posiada jeszcze pełnej i stabilnej implementacji (stan na
początek 2014 roku). Jednakże w niektórych (Chrome, Firefox) API już na tyle dojrzało, że
można go z powodzeniem używać.

\subsection{Web Audio API}
\label{ssec:webAudio}

Poza grafiką, w grach bardzo ważna jest również oprawa dźwiękowa. O ile odtwarzanie muzyki
nie było problemem od czasu wprowadzenia taga <audio> to udźwiękowienie efektów w grze
wymagało bardziej zaawansowanego API.

W tym celu do HTML5 zostało wprowadzone Web Audio API. Jest to wysokopoziomowy, zaawansowany
interfejs do przetwarzania i odtwarzania dźwięków w JavaScript. Umożliwia on wszystko co
potrzebne do udźwiękowienia gry -- dźwięk przestrzenny, nakładanie wielu efektów itp.

Aktualnie (2014) Web Audio API jest wspierane przez wszystkie większe przeglądarki oprócz
Internet Explorer.

\subsection{Pozostałe przydatne API}
\label{ssec:pozostaleApi}

HTML5 zapewnia wiele interfejsów programistycznych przydatnych w wielu aplikacjach
i grach w zależności od potrzeb. Wśród bardziej przydatnych należałoby wymienić:
\begin{itemize}
\item Gamepad API -- obsługa gamepadów, joysticków, kierownic komputerowych itp.
\item Mouse Lock API -- możliwość ukrycia i zablokowania kursora myszy,
\item Fullscreen API
\item Web Storage -- przechowywanie danych po stronie klienta (np. w celu cache'owania danych gry),
\item File API -- operowanie na plikach,
\item XmlHTTPRequest -- asynchroniczne pobieranie plików z serwera (standard de facto
  od dłuższego czasu),
\item Device Orientation i Touch Events -- dla urządzeń mobilnych.
\end{itemize}

%%%%%%%%%%%%%%%%%%%%%%%%%%%%%%%%%%%%%%%%%%%%%%%%%%%%%%%%%%%%%%%%%%%%%%%%%%%%%%%%

\section{Gry Komputerowe}
\label{sec:gryKomputerowe}

Gry komputerowe to w dzisiejszych czasach ogromny rynek zapewniający rozrywkę milionom graczy
na całym świecie. Historia gier zaczęła się niedługo po wynalezieniu komputerów. Początkowo
były to bardzo proste programy, jednak z czasem stawały się coraz bardziej skomplikowane,
zarówno pod względem technicznym (grafika, dźwięk), jak i pod względem mechaniki (zasad gry).

[jakiś skrin?]

Gry opanowały wiele platform sprzętowych. Wcześniej utożsamiane głównie z automatami i konsolami,
wkrótce odniosły wielki sukces na komputerach osobistych,
a współcześnie na telefonach i tabletach.
Co prawda gry działające w przeglądarce pojawiały się od początku istnienia języka JavaScript,
czy platformy Adobe Flash (wcześniej Macromedia Flash), jednak były one utożsamiane
z bardzo prostą rozgrywką i ubogą warstwą audiowizualną. Wraz w nadejściem ery HTML5
i technologii z nim związanych (przede wszystkim WebGL), zaistniała możliwość ekspansji
bardziej zaawansowanych gier na ten -- stosunkowo jeszcze młody -- rynek.

W następnych podrozdziałach zostanie przedstawiony gatunek gier FPS oraz dwie gry tego typu,
szczególnie związane z niniejszym opracowaniem, a także silnik gry, na którym są oparte.
Następnie wprowadzone zostaną koncepcje związane z architekturą gier sieciowych (mutliplayer).

\subsection{Gry FPS}

Przedmiotem pracy jest stworzenie gry typu FPS działającej w przeglądarce. 

Skrót FPS oznacza First Person Shooter. Jest to strzelanina w której obserwujemy świat
oczami bohatera. Gry typu FPS należą do najstarszych i najczęściej występujących gier
3d. Z jednej strony mają one dość prostą mechanikę i zasady, z drugiej natomiast, często
są bardzo zaawansowane technologicznie i bywają pionierskie w dziedzinach takich jak
realistyczna grafika, czy rozgrywka sieciowa. Dlatego właśnie taki gatunek został wybrany
do implementacji w HTML5.

[screen ze współczesnej gry]

\subsection{Quake III Arena}
\label{ssec:quake3}

W roku 1996 ukazała się gra Quake, która odniosła wielki sukces, głównie dzięki
rewolucyjnej grafice i dopracowanej rozgrywce sieciowej. Twórcą gry była firma
id Software. Rok później powstała kolejna część - Quake II.

Z racji ogromnej popularności trybu dla wielu graczy, kolejna część -- Quake III
Arena -- zupełnie porzuciła tradycyjny tryb fabularny i skupiła się tylko na
rozgrywce sieciowej. Została wydana w roku 1999.
Gra również odniosła sukces i do dziś pozostaje popularna
pośród graczy. Profesjonalne turnieje gier sieciowych (tzw. e-sport) bardzo
często w swoim programie mają rozgrywki w Quake III Arena.
W roku 2000 ukazał się dodatek do gry pod nazwą Quake III Team Arena.

[screen]

Siedem lat później, w roku 2007 id Software rozpoczął prace nad wydaniem
Quake III Arena i Team Arena na przeglądarki internetowe pod nazwą Quake Live.
Kod gry uruchamiany był poprzez specjalnie stworzoną wtyczkę do
przeglądarki. W związku z tym rozgrywka była możliwa tylko na wybranych systemach
operacyjnych i przeglądarkach, konieczna też była instalacja wtyczki przez
gracza.

w międzyczasie, w roku 2005 ukazał się Quake IV, który kontynuował wątki fabularne
z Quake II. Gra pomimo dobrych ocen w prasie nie przyjęła się tak dobrze
wśród graczy, głównie ze względu na brak znaczących ulepszeń w trybie sieciowym.

Quake III oraz Quake Live jest oparty na silniku gry o nazwie id Tech 3, o którym
będzie mowa w podrozdziale \ref{ssec:idTech}.

\subsection{OpenArena}
\label{ssec:openArena}

W roku 2005 główny programista id Software, John Carmack, ogłosił uwolnienie
kodu źródłowego silnika id Tech 3 na licencji GNU GPL. Uwalnianie źródeł swoich
starszych technologii jest już tradycją w tej firmie.

Wkrótce rozpoczęły się prace nad wieloma grami open source korzystającymi
z tego silnika. Jedną z najbardziej dojrzałych jest OpenArena.

OpenArena jest właściwie klonem Quake III Arena. Ponieważ z pierwowzoru został
udostępniony tylko kod, fani utworzyli potrzebne zasoby gry (modele, tekstury itp.),
również na licencji GPL. Wprowadzono też wiele usprawnień do kodu gry,
dodając np. obsługę nowszej wersji OpenGL i shaderów \footnote{Shader - niewielki program
  działający na karcie graficznej, opisujący właściwości wierzchołków
  i pikseli.}. Tym samym OpenArena stała się pełnoprawną
i całkowicie otwartą grą.

[skrin]

\subsection{id Tech 3}
\label{ssec:idTech}

Każda bardziej zaawansowana gra posiada w kodzie źródłowym mniej lub bardziej
rozgraniczony rdzeń nazywany silnikiem gry. Odpowiada on za wczytywanie
i wyświetlanie świata 3d, odtwarzanie dźwięków, obsługę sieci oraz wiele
innych zadań. Często jeden silnik jest wykorzystywany przy tworzeniu wielu gier.

id Software zawsze wykorzystywało swoje autorskie silniki, których źródła były
publikowane po utworzeniu nowej wersji. Quake III Arena, a potem OpenArena
wykorzystywały id Tech w wersji 3.

Silnik ten był bardzo zaawansowany technologicznie jak na swoje czasy. Do wyświetlania
grafiki konieczny był sprzętowy akcelerator graficzny. Pośród ciekawszych
możliwości tego silnika należy wymienić:
\begin{itemize}
\item wyświetlanie powierzchni opartych na krzywych sklejanych,
\item system tzw. skryptów shaderów, dzięki którym graficy mieli dużą kontrolę
  nad wyglądem każdej powierzchni \footnote {Nie należy mylić tego systemu nieistniejącymi
    jeszcze shaderami działającymi na karcie graficznej. Skrypty z id Tech3 były jednak pewną
    namiastką shaderów w dzisiejszym rozumieniu.},
\item wyświetlanie generowanych wcześniej map oświetlenia, zapewniających bardziej
  realistyczny wygląd otoczenia,
\item przestrzenna mgła,
\item odbicia lustrzane,
\item zaawansowany system synchronizacji stanu gry przez sieć,
\item QVM -- wbudowana wirtualna maszyna wykonująca kod gry.
\end{itemize}

\subsection{Mulltiplayer}

Multiplayer jest to tryb gry, w którym uczestniczy wielu graczy, zazwyczaj za
pośrednictwem Internetu. Tryb ten jest często spotykany w grach FPS, ale także
w grach wyścigowych, czy strategiach. Wiele gier i graczy skupia się wyłącznie
na grze sieciowej. Zdarza się, że w jednej rozgrywce uczestniczy na raz
tysiące graczy \footnote {Są to tak zwane gry MMO -- Massive Multiplayer Online}

Tryb multiplayer wymaga, aby wszyscy gracze widzieli ten sam stan gry.
Synchronizacja gry pomiędzy komputerami w sieci jest zagadnieniem nietrywialnym,
szczególnie w przypadku gier bardzo dynamicznych i posiadających złożoną mechanikę
(a przez to złożony opis stanu gry).

Z punktu widzenia organizacji w sieci, możliwe są dwie architektury:
\begin{itemize}
\item Peer-to-peer -- nie ma głównego komputera, każdy komputer rozsyła do wszystkich
  pozostałych informacje o każdej akcji wykonanej przez gracza lokalnego. Architektura
  ta jest rzadko używana, ze względu na trudność z synchronizacją i problemy z
  nieuczciwymi graczami.
\item Klient - serwer -- najczęściej używana. Jeden z komputerów jest serwerem
  gry. Na nim odbywa się cała symulacja, a wszyscy klienci służą jedynie jako
  terminale, wysyłając informacje o naciśniętych przez gracza przyciskach
  i otrzymując migawkę (snapshot) aktualnego stanu gry do wyrenderowania.
\end {itemize}

[obrazek z porównaniem]

Należy zaznaczyć, że serwer niekoniecznie musi być dedykowaną maszyną utrzymywaną
przez producenta gry. W przypadku gier w niewielką liczbą uczestników,
zazwyczaj jeden z komputerów graczy pełni rolę serwera. Dodatkowo, bardzo
często producent zapewnia tzw. lobby server, czyli główny serwer,
który służy m. in. do odnajdywania toczących się właśnie rozgrywek
i inicjalizacji połączenia. W związku z tym pojęcie serwer może mieć
dwa różne znaczenia i wymaga doprecyzowania, dlatego na potrzeby dalszej części pracy
wprowadzimy następujące pojęcia:
\begin{itemize}
\item serwer główny lub lobby -- serwer utrzymywany przez dewelopera, zbierający
  informacje o toczących się grach i ułatwiający inicjalizację połączenia,
\item serwer gry -- instancja gry która zarządza konkretną rozgrywką
\end{itemize}
W przypadku gry przeglądarkowej, serwer gry działa w przeglądarce, co może
wprowadzić dodatkowe nieporozumienie, gdyż w aplikacjach webowych przeglądarka
jest tożsama z klientem.

%%%%%%%%%%%%%%%%%%%%%%%%%%%%%%%%%%%%%%%%%%%%%%%%%%%%%%%%%%%%%%%%%%%%%%%%%%%%%%%%

\section{Poprzednie prace}
\label{sec:poprzedniePrace}

W niniejszym rozdziale zaprezentowane zostaną projekty, mające związek z tematyką pracy.

\subsection{Quake II w przeglądarce}
\label{ssec:quake2web}

W roku 2010 dwaj pracownicy Google, Joel Webber i Ray Cromwell
udostępnili swój projekt -- port Quake II
na przeglądarki z wykorzystaniem WebGL. Była to jedna z pierwszych prac,
które udowodniły możliwość zaistnienia gier 3d w przeglądarkach.

quake2-gwt-port, bo tak został nazwany projekt, powstał poprzez skompilowanie
w GWT [tu ref/footnote/whatever] kodu Jake2 -- portu Quake II do Javy.
Projekt ten nie powstał bezpośrednio w JavaScript, przez co nie integruje się
zbyt dobrze ze środowiskiem webowym, a do uruchomienia konieczne było
wiele łat i rozwiązań prowizorycznych. Dodatkowo w roku 2010 nie istniało
jeszcze Mouse Lock API, przez co sterowanie grą jest bardzo trudne.

Z tych powodów projekt ten należy uznać raczej za demo technologiczne niż
grę. Mimo to był to ważny krok milowy dla technologii WebGL.

[strona projektu]

\subsection{Quake III level viewer}
\label{ssec:quake3web}

W tym samym roku Brandon Jones zaprezentował swoje demo o nazwie q3bsp.
Jest to level viewer\footnote{Level viewer - narzędzie umożliwiające
  podgląd danego poziomu gry, bez uruchamiania logiki rozgrywki} poziomów
z Quake III Arena. W przeciwieństwie do quake2-gwt-port, q3bsp został
w całości napisany w JavaScripcie, dzięki czemu mógł użyć wiele udogodnień
zapewnianych przez środowisko przeglądarki (np. zapewnione wczytywanie
plików graficznych, interfejs użytkownika w HTML), a jego wydajność
jest dużo lepsza, pomimo bardziej zaawansowanej grafiki.

Projekt ten również zyskał duży rozgłos w środowisku deweloperów WebGL
i poza nim. Można go zobaczyć na stronie [strona]. Część kodu źródłowego
q3bsp (udostępnionego na licencji zlib) została wykorzystana w niniejszej pracy.

\subsection{Banana Bread}
\label{ssec:bananaBread}

Banana Bread jest portem gry Cube2:Sauerbraten [link] na JavaScript zrobionym
przez Mozillę. Powstał przez kompilację z C++ kompilatorem Emscripten.

Jest to pełnoprawna i działająca gra, w którą można zagrać na stronie [ref].


%%% Local Variables: 
%%% mode: latex
%%% TeX-master: "praca"
%%% End: 
