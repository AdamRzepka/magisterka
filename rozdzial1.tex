\chapter{Wprowadzenie}
\label{cha:wprowadzenie}

Standard HTML 5 otwiera przed twórcami aplikacji internetowych nowe możliwości. Technologie z nim
związane, takie jak WebGL, spowodowały, że również autorzy zaawansowanych gier komputerowych
zaczęli przyglądać się platformie internetowej, w nadziei znalezienia nowego rynku dla swoich
produktów.

Niniejsza praca jest próbą zbadania tego tematu przez osobę należącą do branży twórców gier.
Podczas tworzenia gry w technologii HTML 5 zostanie położony nacisk na porównanie z tradycyjnymi
grami, a następnie podsumowanie zalet i wad nowego podejścia.

%---------------------------------------------------------------------------

\section{Cele pracy}
\label{sec:celePracy}

Praca ma następujące cele:
\begin{enumerate}
\item Stworzenie gry typu FPS z wykorzystaniem technologii zawartych w HTML 5.
\item Analiza możliwości tych technologii w porównaniu do tradycyjnego podejścia do tworzenia gier.
\end{enumerate}

% ---------------------------------------------------------------------------

\section{Teza pracy}
\label{sec:teza}

Teza pracy:

\emph{Technologia HTML 5 umożliwia tworzenie zaawansowanych gier 3D.}

\section{Zawartość pracy}
\label{sec:zawartoscPracy}

W następnym rozdziale zostaną omówione kwestie związane z tematem pracy. W pierwszej części
rozdziału następuje przedstawienie HTML 5 i związanych z nim technologii istotnych z punktu
widzenia tworzenia gry. Następnie będą wyjaśnione pojęcia związane z grami komputerowymi --
gry typu FPS, silnik gry, czy tryb multiplayer. Rozdział kończy się zaprezentowaniem
kilku poprzednich prób związanych z grami HTML 5.

Rozdział \ref{cha:projekt} to założenia co do tworzonego projektu oraz omówienie budowy jako
gry komputerowej oraz aplikacji sieciowej.

W rozdziale \ref{cha:implementacja} znajduje opis implementacji gry, z omówieniem poszczególnych
komponentów. W tym rozdziale zostaje też krótko przedstawiona gra z perspektywy końcowego
użytkownika.

Kolejny rozdział prezentuje analizę stworzonego projektu. Skupia się przede wszystkim na różnicach
w procesie tworzenia gry HTML 5 i gry działającej natywnie. Wymienia zalety i wady nowego podejścia.

Ostatni rozdział to podsumowanie pracy z oceną realizacji celów i prawdziwości tezy pracy.












%%% Local Variables: 
%%% mode: latex
%%% TeX-master: "praca"
%%% End: 
