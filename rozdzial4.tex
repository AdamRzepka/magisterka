\chapter{Analiza}
\label{cha:analiza}

Niniejszy rozdział zaczyna się od  podsumowania zrealizowanego projektu. Następnie na podstawie zebranych
doświadczeń nastąpi analiza przydatności technologii HTML 5 do realizacji gier 3d.

\section{Stopień realizacji projektu}
\label{sec:stopieńRealizacji}

Realizowany projekt -- WebArena -- miał być okrojonym klonem gry OpenArena, działającym w środowisku webowym.

O ile gra jest w pewnym stopniu uproszczona w porównaniu do pierwowzoru, o tyle udało się odtworzyć
większość funkcjonalności silnika gry, a więc dodawanie kolejnych funkcjonalności na poziomie samej
rozgrywki jest już zadaniem stosunkowo prostym (choć wymaga trochę czasu) i nie wpłynie w znacznym stopniu
na wydajność gry czy jej ogólną konstrukcję.

Największym brakiem w grze jest brak dźwięków. Nie wynika on jednak z niedostatków technologicznych
HTML 5, gdyż standard oferuje bardzo rozbudowaną bibliotekę do obsługi dźwięku (rozdział \ref{ssec:webAudio},
a jedynie z ograniczonych zasobów czasowych.

Największym wyzwaniem była realizacja renderera i multiplayera. Jednocześnie są to najważniejsze komponenty
dla sieciowej gry 3D. W WebArana zostały one odtworzone z prawie całą funkcjonalnością dostępną
w pierwowzorze. Renderer ma następujące braki:
\begin{itemize}
\item brak wyświetlania mgły i wody,
\item brak wyświetlania śladów po kulach na ścianach,
\item brak luster,
\item brak optymalizacji polegającej na nierenderowaniu odległych przedmiotów.
\end{itemize}

Dodanie tych funkcjonalności nie będzie trudnym zadaniem przy istniejących już podstawach, a poza ostatnim
punktem nie wpływają one znacząco na wydajność gry.

Jeśli chodzi o część odpowiedzialną za multiplayer, to jedynym poważniejszym brakiem jest przyjazna dla
użytkownika obsługa utraty połączenia sieciowego.

\section{Porównanie z grami natywnymi}
\label{sec:porownanieZNatywnymi}

\subsection{Wydajność}

Największą obawą deweloperów gier co do gier przeglądarkowych jest wydajność języka JavaScript w stosunku
do C i C++. W najbardziej zaawansowanych graficznie grach każdy niemalże cykl procesora jest ważny.
W ich przypadku języki natywne są jedynym możliwym wyborem. Tego typu gry to jednak niewielki ułamek
całego rynku.

Od kilku lat trwa wyścig pomiędzy twórcami przeglądarek o najszybszy silnik JavaScript. W rezultacie,
aktualnie JavaScript jest prawdopodobnie najszybszym językiem skryptowym. Jest to osiągnięte poprzez
kompilację do kodu natywnego przed wykonaniem, profilowanie w czasie rzeczywistym i optymalizację
najczęściej wykonywanego kodu, cache'owanie informacji o typie oraz bardzo szybkie odśmiecanie.

W efekcie, podczas tworzenia WebArana nie pojawił się ani raz problem ze zbyt niską wydajnością języka.
Gra na średniej klasy komputerze sprzed kilku lat działa płynnie, a możliwych jest 
wiele optymalizacji w rendererze, które jeszcze przyspieszą działanie gry. 

Istotny jest również sposób realizacji API WebGL. Jest ono bardzo niskopoziomowe i bardzo podobne
do natywnego OpenGL. Shadery wykonują się na procesorze karty graficznej dokładnie tak samo jak
w przypadku OpenGL. Dzięki temu wydajność WebGL w większości wypadków jest praktycznie taka sama jak
OpenGL.

Podczas tworzenia gry trzeba jednak zwracać uwagę na kwestię istnienia garbage collectora. Pomimo,
że te zaimplementowane we współczesnych przeglądarkach są bardzo szybkie, to jeśli aplikacja
będzie alokować pamięć bardzo intensywnie, może się okazać, że odśmiecanie powoduje zauważalną pauzę.
Z tego powodu najlepiej alokować większe ilości pamięci podczas wczytywania gry, a w czasie działania
ograniczyć tworzenie nowych obiektów do minimum.

Należy zaznaczyć, że przekazane w tej sekcji dotyczą przede wszystkim
przeglądarek Google Chrome i Mozilla Firefox, gdyż
aktualnie tylko na nich działa gra (ze względu na implementacją potrzebnych standardów). Można się jednak
spodziewać, że pozostałe przeglądarki zaoferują podobną wydajność silników skryptowych.

\begin{figure}[h]
  \centering
  \includegraphics[scale=1]{zasoby/rozdzial4/fps}
  \caption {Średnia ilość klatek na sekundę w grze}
  \label{fig:fps}
\end{figure}

Na wykresie \ref{fig:fps} przedstawiono średnią ilość klatek na sekundę\footnote{FPS (frames per second)
  -- ilość klatek wyrenderowanych podczas jednej sekundy. Popularna metryka wydajności gry.}. Jak
widać, Google Chrome jest szybsze, jednak Firefox również zdołał wygenerować około 40 klatek co jest
wynikiem wystarczającym do komfortowej gry. Trzeba zwrócić uwagę, że przy standardowych
ustawieniach przeglądarki, sterownik karty graficznej i tak ograniczy ilość klatek do 60, w celu
synchronizacji z częstotliwością odświeżania monitora. Końcowy użytkownik nie zauważy więc tak dużej różnicy
w wydajności pomiędzy przeglądarkami, jak widać to na porównaniu.

\subsection{Wielowątkowość}

Tak jak napisano w rozdziale \ref{ssec:webWorkers}, wielowątkowość jest realizowana w JavaScript przez API
WebWorkers, które nakłada wiele ograniczeń. Workery są bardziej podobne do procesów niż wątków, ponieważ
nie mogą współdzielić danych, ani kodu. Dodatkowo większość funkcji JavaScript jest niedostępnych
poza głównym workerem (np. WebGL, WebRTC, Web Audio API), musi on więc pośredniczyć we wszystkich
wywołaniach API.

Taka konstrukcja sprawia, że bardzo trudno jest uzyskać znaczący wzrost wydajności dzięki rozdzieleniu
pracy na np. dwa wątki. W przypadku WebArena można to było łatwo zauważyć dzięki bibliotece opisanej
w rozdziale \ref{sec:wielowatkowosc}, która umożliwia szybkie przełączenie pomiędzy trybem jednowątkowym
i dwuwątkowym. W pierwszym trybie, renderer i logika działają na wątku głównym, w drugim zaś są podzielone
na dwa wątki.

Obserwacje wykazały, że wydajność gry, mierzona ilością wyświetlonych klatek na sekundę, zmienia
się w niewielkim stopniu podczas przełączania trybów.

\begin{figure}[h]
  \centering
  \includegraphics[scale=1]{zasoby/rozdzial4/fpsthreads}
  \caption {Porównanie ilość klatek na sekundę w trybie jednowątkowym i dwuwątkowym}
  \label{fig:fpsThreads}
\end{figure}

Wykres \ref{fig:fpsThreads} przedstawia porównanie wydajności tych trybów. W Firefoksie udało się osiągnąć
pewien zysk rozdzielając pracę na dwa wątki, jednak jest to zysk niewielki. Jedną z przyczyn takiej sytuacji
jest to, że w celu komunikacji pomiędzy workerami konieczne jest kopiowanie dużych ilości danych, które nie mogą
być współdzielone. To z kolei powoduje częstsze wywoływanie garbage collectora, co niweluje część
zysków wynikających ze zrównoleglenia.

W Chrome nie występuje zauważalna różnica w wydajności. Jest to prawdopodobnie spowodowane tym, że silnik
JavaScript w tej przeglądarce jest już na tyle szybki, że wąskim gardłem staje się procesor graficzny.
Wobec tego przyśpieszanie kodu działającego na procesorze głównym, nie może zwiększyć ogólnych wyników.

Przy tych pesymistycznych wnioskach trzeba jednak zwrócić na to, że WebArena nie ma zbyt skomplikowanej
logiki, przez co główny wątek (renderujący) ma więcej pracy niż wątek logiki gry. W takiej sytuacji
niemożliwe jest bardzo duże przyśpieszenie przez podział na dwa wątki.

Gry w których występuje np. symulacja fizyki i zaawansowana animacja szkieletowa, wciąż warto jest
stosować Web Workers, najlepiej z dużą ilością krótkich zadań do wykonania, które łatwo oddelegować
na dowolną ilość wątków. Takie rozwiązanie przyniosłoby dobre efekty zwłaszcza na procesorach posiadających
więcej niż dwa rdzenie. Jeżeli jednak logika gry jest prosta, koszt projektowania
systemu wielowątkowego specjalnie pod Web Workers, prawdopodobnie nie jest warty potencjalnych zysków.

Należy też zwrócić uwagę na jeszcze jedną zaletę Web Workers. Nawet jeżeli dodatkowe workery nie przyniosą
zysków wydajnościowych, to umożliwią wykonywanie czasochłonnych zadań (takich jak parsowanie pliku
binarnego) bez blokowania interfejsu użytkownika. Jest to bardzo istotne patrząc od strony ergonomii
użytkowania gry.

\subsection{Proces programowania gry}

Tworzenie gry przeglądarkowej z punktu widzenia programisty wygląda inaczej niż w przypadku natywnych
gier. JavaScript jest językiem skryptowym, uruchamianym w przeglądarce.

Ma to następujące zalety:
\begin{itemize}
\item JavaScript jest bardzo elastyczny i programuje się w nim łatwiej niż np. w C.
\item Brak konieczności kompilacji sprawia, że bardzo szybko można wprowadzać i testować
  poprawki.
\item Możliwość edycji kodu w przeglądarce, co jeszcze bardziej przyśpiesza testowanie
  poprawek.
\item Dobre narzędzia do debugowania kodu JavaScript oraz błędów w wyświetlaniu grafiki 3d.
\end{itemize}

Z kolei do wad należą:
\begin{itemize}
\item Mniejsza wydajność w porównaniu z językami natywnymi (aczkolwiek w większości przypadków
  wystarczająca),
\item Brak kompilacji powoduje, że później zostają wykryte błędy,
\item Elastyczność JavaScriptu w rękach niedoświadczonego programisty może prowadzić
  do powstania źle zorganizowanego i trudnego w utrzymaniu kodu.
\item Mniejsza dostępność gotowych bibliotek wspomagających tworzenie gier.
\end{itemize}

\subsection{Środowisko przeglądarkowe}

Fakt, że gra działa wewnątrz strony internetowej powoduje, że wiele problemów, koniecznych
do rozwiązania podczas pisania gry natywnej, po prostu nie istnieje.

Jeden z przykładów to parsowanie plików graficznych. Jeżeli tekstury są dostarczane w jednym
z formatów obsługiwanych przez przeglądarkę, nie trzeba pisać dodatkowego parsera, gdyż
przeglądarka zajmie się ich wczytaniem. Obrazek taki można następnie przekazać do WebGL
jako teksturę lub po prostu wyświetlić jako element interfejsu, również bez konieczności
tworzenia własnego renderera 2D.

Podobnie jest z plikami dźwiękowymi, czy filmami występującymi w grach jako wprowadzenie
w fabułę.

Jeszcze większą zaletą osadzenia gry w środowisku strony internetowej, jest dostęp do
drzewa dokumentu HTML. Dzięki temu programista ma do dyspozycji gotowe rozwiązanie
do tworzenia interfejsu użytkownika. W dodatku jest to jedno z najbardziej przemyślanych
i wypróbowanych rozwiązań, a jednocześnie jedno z najpotężniejszych pod względem oferowanych
możliwości.

Należy tutaj zaznaczyć, że stworzenie dobrego graficznego interfejsu użytkownika w grach jest
trudnym i żmudnym zadaniem. Już samo prawidłowe renderowanie tekstów jest zadaniem nietrywialnym.
Mając do dyspozycji HTML, problem ten znika.

\subsection{Wieloplatformowość}

Twórcy gier dążą do tego, aby ich produkty działały na jak największej ilości urządzeń. Jednak
proces portowania gry na nową platformę jest bardzo często trudny i nieopłacalny. W przypadku
HTML 5 sprawa jest prosta -- jeżeli przeglądarka na danej platformie oferuje wszystkie potrzebne
API, to gra po prostu będzie działać.

Już teraz WebGL jest dostępne na wszystkich liczących się platformach PC (Windows, Mac, Linux),
oraz wielu mobilnych. Należy założyć, że również przeglądarki działające na konsolach do gier nowej
generacji zapewnią wkrótce obsługę tego standardu, co otworzy możliwość dotarcia do ich użytkowników
dla niezależnych deweloperów\footnote{Gry na konsole takie jak PlayStation czy Xbox mogą tradycyjnie tworzyć
tylko certyfikowani deweloperzy.}.

Oczywiście samo uruchomienie gry to nie wszystko. Często trzeba dostosować sposób kontroli w grze
(np. dla ekranów dotykowych), czy wyświetlania na mniejszym ekranie. Mimo to HTML 5 jest dużym ułatwieniem
dla tworzenia gier wieloplatformowych.

\subsection{Marketing i Monetyzacja}

Aby sprzedać grę należy dotrzeć do jak największej liczby odbiorców. Tutaj platforma internetowa pokazuje
swoje kolejne
zalety. Jedną z nich jest łatwa integracja z serwisami społecznościowymi, które są kluczowe
dla zaistnienia w świadomości użytkowników.

Kolejną istotną cechą jest łatwa dostępność gry. Nie ma potrzeby czasochłonnego pobierania i instalacji.
Wystarczy kliknięcie w link i po chwili można zacząć rozgrywkę.

W tym miejscu trzeba jednak napisać o prawdopodobnie aktualnie największym problemie gier HTML 5.
Jest nim trudność monetyzacji gry wynikająca z tego, że rynek dopiero zaczyna się rozwijać.

Dla gier działających natywnie na PC istnieje wiele sklepów i platform dystrybucji cyfrowej. Są one
bardzo popularne wśród graczy, można więc za ich pośrednictwem łatwo trafić do klientów
zainteresowanych płaceniem za gry. Dodatkowo sklep zajmuje się obsługą transakcji.

Dla gier webowych takie platformy dopiero się pojawiają i na razie mają niewielką liczbę klientów.
Dlatego aktualnie najlepszym sposobem na zarobienie na grze jest model free to play. W tego typu
gry każdy może zacząć grać za darmo i płacić wedle uznania za dodatkowe usługi. Niestety taki
model nie nadaje się do każdej gry.

Jeden z serwisów który może w przyszłości zmienić tę sytuację jest \url{turbulenz.com}.
Oferuje on już kilka darmowych gier,
pracuje także nad wprowadzeniem płatności. Co więcej, oferuje też darmowy, kompletny silnik
do tworzenia gry w HTML 5, który można wykorzystać również poza serwisem.

\section{Podsumowanie przydatności HTML5 do tworzenia gier}
\label{sec:przydatność}

W tym podrozdziale znajduje się krótkie podsumowanie wymienionych wcześniej zalet i wad
tworzenia gier w HTML 5.

\subsection{Zalety}
\label{ssec:zalety}

\begin{itemize}
\item Łatwość i szybkość tworzenia kodu w JavaScript.
\item Bogate API HTML (np. do tworzenia interfejsu użytkownika).
\item Narzędzia ułatwiające tworzenie aplikacji internetowych.
\item Wieloplatformowość.
\item Łatwa integracja z serwisami społecznościowymi.
\item Natychmiastowa dostępność dla graczy.
\end{itemize}

\subsection{Wady}
\label{ssec:wady}

\begin{itemize}
\item Mniejsza wydajność JavaScript w porównaniu do języków natywnych.
\item Brak tradycyjnych wątków.
\item Mniejsza dostępność bibliotek wspomagających tworzenie gier.
\item Elastyczność i brak kompilacji JavaScript może stać się wadą.
\item Trudność monetyzacji.
\end{itemize}



%%% Local Variables: 
%%% mode: latex
%%% TeX-master: "praca"
%%% End: 
