\chapter{Podsumowanie}
\label{cha:podsumowanie}

Nowe technologie związane z HTML 5 stwarzają dla twórców gier szansę zdobycia nowych rynków.
Niniejsza praca miała na celu zbadanie tematu tworzenia gier w HTML 5, na przykładzie gry
typu FPS.
Podczas pisania pracy udało się stworzyć projekt, odtwarzający jedną ze znanych gier
w przeglądarce internetowej.


\section{Ocena realizacji celów pracy}
Praca miała następujące cele:
\begin{enumerate}
\item Stworzenie gry typu FPS z wykorzystaniem technologii zawartych w HTML 5.
\item Analiza możliwości tych technologii w porównaniu do tradycyjnego podejścia do tworzenia gier.
\end{enumerate}

Cele te udało się zrealizować.

Realizację celu pierwszego można śledzić w rozdziałach \ref{cha:projekt} i \ref{cha:implementacja}.
Cel drugi został zrealizowany w rozdziale \ref{cha:analiza}.


\section{Ocena prawdziwości tezy pracy}

Niniejsza praca dowodzi prawdziwości tezy:
\emph{Technologia HTML 5 umożliwia tworzenie zaawansowanych gier 3D.}

\section{Możliwości dalszego rozwoju projektu}
\label{sec:mozliwosciRozwoju}

O ile projekt realizuje badawczy cel pracy, o tyle prezentuje raczej poziom technologicznego
eksperymentu niż produktu komercyjnego. Możliwe jest jednak rozbudowanie go, aby nadawał się
do upublicznienia w internecie. Oto potencjalne kierunki rozwoju gry:

\begin{itemize}
\item ulepszenie i optymalizacja renderera,
\item dodanie nowych broni i przedmiotów do gry,
\item nowe poziomy,
\item nowe tryby gry (np. drużynowy),
\item nowe modele postaci,
\item integracja z portalami społecznościowymi,
\item rankingi graczy,
\item system zabezpieczający przed oszukiwaniem w rozgrywce sieciowej,
\item wprowadzenie opcjonalnych płatności za dodatkową zawartość w grze.
\end{itemize}



%%% Local Variables: 
%%% mode: latex
%%% TeX-master: "praca"
%%% End: 
